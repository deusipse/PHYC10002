\documentclass[a4paper]{scrartcl}

\usepackage{mathtools}
\usepackage{amsthm}
\usepackage{amsfonts}
\usepackage{parskip}
\usepackage{microtype}
\usepackage{siunitx}
\usepackage{hyperref}

\newtheorem{theorem}{Theorem}
\newtheorem{axiom}{Axiom}
\newtheorem{lemma}{Lemma}
\newtheorem{proposition}{Proposition}
\theoremstyle{definition}
\newtheorem{example}{Example}
\newtheorem{definition}{Definition}
\newtheorem{remark}{Remark}

\DeclarePairedDelimiter\abs{\lvert}{\rvert}
\DeclarePairedDelimiter\norm{\lVert}{\rVert}
\DeclarePairedDelimiter\ceil{\lceil}{\rceil}
\DeclarePairedDelimiter\floor{\lfloor}{\rfloor}
\DeclarePairedDelimiterX\inpr[2]{\langle}{\rangle}{#1, #2}

\newcommand\C{\mathbb{C}}
\newcommand\R{\mathbb{R}}
\newcommand\Q{\mathbb{Q}}
\newcommand\Z{\mathbb{Z}}
\newcommand\N{\mathbb{N}}
\renewcommand\O{\varnothing}
\let\epsilon\varepsilon
\let\vec\mathbf

\begin{document}
\title{Physics 2: Advanced \\ PHYC10002}
\author{Edward Wang}
\date{Semester 2, 2025}
\maketitle

\tableofcontents

\section{Electricity}
\begin{definition}
  \emph{Electrostatics} concerns forces between charges at rest.
\end{definition}
\begin{theorem}[Coulomb's law]
  The electrostatic force experienced by a charge $q_1$ in the vicinity of another charge $q_2$ is equal to \[
    \vec{F}_{1} = \frac{1}{4\pi \epsilon_0} \frac{q_1q_2}{\abs{\vec{r}_{12}}^2}\hat{\vec{r}}_{12},
  \] 
  where $\epsilon_0 \approx \qty{8.85e-12}{C^2.N^{-1}.m^{-2}}$.
\end{theorem}
\begin{definition}[Coulomb's constant]
  Coulomb's law is sometimes written as \[
    \vec{F}_{1} = k_e\frac{q_1q_2}{\abs{\vec{r}_{12}}^2}\hat{\vec{r}}_{12},
  \] 
  where $k_e \approx \qty{8.99e9}{N.m^2/C^2}$.
\end{definition}
\begin{remark}
  The electromagnetic force at a nuclear scale is far stronger than the gravitational force. Consider an electron and a proton about $\qty{e-10}{m}$ apart. Given that $\qty{8.99e9}{N.m^2/C^2}$, $G\approx \qty{6.67e-11}{m^3.kg^{-1}.s^{-2}}$, $m_e \approx \qty{9.1e-31}{kg}$, $m_p \approx \qty{1.6e-27}{kg}$, and $e \approx \qty{1.6e-19}{C}$, we would have  \[
    \abs{\vec{F}_E} = k_e\frac{q_1q_2}{r^2} \approx 2.3 \times 10^{8} \gg 9.7 \times 10^{-48} \approx G \frac{m_1m_2}{r^2} = \abs{\vec{F}_g}.
  \] 
  It should also be noted that the strong nuclear force, the force of the gluons binding the quarks together within nucleons, is far stronger than the electromagnetic force.
\end{remark}


\end{document}
